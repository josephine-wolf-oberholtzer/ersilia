\documentclass[10pt]{article}
\usepackage{enumitem}
\usepackage[utf8]{inputenc}
\usepackage[papersize={8.5in, 14in}]{geometry}
\usepackage[absolute]{textpos}
\TPGrid[0.5in, 0.25in]{23}{24}
\usepackage{palatino}
\parindent=0pt
\parskip=12pt
\usepackage{nopageno}

\begin{document}

\begin{textblock}{23}(0, 0)
\center\huge PREFACE
\end{textblock}

\begin{textblock}{9}(0, 2)

\section{}

\textit{In Ersilia, to establish the relationships that sustain the city's
life, the inhabitants stretch strings from the corners of the houses, white or
black or gray or black-and-white according to whether they mark a relationship
of blood, of trade, authority, agency. When the strings become so numerous that
you can no longer pass among them, the inhabitants leave: the houses are
dismantled; only the strings and their supports remain.}

\textit{From a mountainside, camping with their household goods, Ersilia's
refugees look at the labyrinth of taut strings and poles that rise in the
plain. That is the city of Ersilia still, and they are nothing.}

\textit{They rebuild Ersilia elsewhere. They weave a similar pattern of strings
which they would like to be more complex and at the same time more regular than
the other. Then they abandon it and take themselves and their houses still
farther away.}

\textit{Thus, when traveling in the territory of Ersilia, you come upon the
ruins of abandoned cities, without the walls which do not last, without the
bones of the dead which the wind rolls away: spiderwebs of intricate
relationships seeking a form.}

- Italo Calvino, \emph{Invisible Cities}

\section{}

\textit{"Are we still in the South Reach?"}

\textit{"Reach? No. The islands --" The chief moved his slender black hand in
an arc, no more than a quarter of the compass, north to east. "The islands are
there," he said. "All the islands." Then showing all the evening sea before
them, from north through west to south, he said, "The sea."}

\textit{"What land are you from, lord?"}

\textit{"No land. We are the Children of the Open Sea."}

\textit{Arren looked at his keen face. He looked about him at the great raft
with its temple and its tall idols, each carved from a single tree, great
god-figures mixed of dolphin, fish, man, and seabird; at the people busy at
their work, weaving, carving, fishing, cooking on raised platforms, tending
babies; at the other rafts, seventy at least, scattered out over the water in a
great circle perhaps a mile across. It was a town: smoke rising in thin wisps
from distant houses, the voices of children high on the wind. It was a town,
and under its floors was the abyss.}

- Ursula LeGuin, \emph{The Farthest Shore}

\end{textblock}

\begin{textblock}{13}(10, 2)

\section{Instrumentation}

\begin{itemize}
    \item[-] Flute
    \item[-] Bass clarinet
    \item[-] Oboe
    \item[-] Baritone saxophone
    \item[-] Acoustic guitar
    \item[-] Piano
    \item[-] Percussion
        \begin{itemize}
            \item[-] bamboo wind chimes
            \item[-] four toms
            \item[-] five wood blocks
            \item[-] snare drum
            \item[-] marimba
            \item[-] crotales, two octaves
            \item[-] tam-tam
            \item[-] bass drum
        \end{itemize}
    \item[-] Violin
    \item[-] Viola
    \item[-] Cello
    \item[-] Contrabass
\end{itemize}

\section{Performance notes}

Six players -- flute, bass clarinet, oboe, violin, viola and cello -- receive a
shaker -- caxixi, maraca or similar. Four players -- guitar, piano, percussion
and contrabass -- receive a chromatic pitch pipe -- a circular harmonic-like
instrument generally used for tuning vocal groups.

The shakers should be placed on or suspended from their respective performer's
music stands, or wherever convenient.

The pitch-pipes should be played by inhaling or exhaling -- as indicated --
through fully half of the circumference of the instrument, creating a rich
cluster.

For all winds, a +-symbol indicates slap tonguing.

For guitar, the coda symbol indicates a percussive damping of the strings.

Piano plays with pedal to their discretion throughout the first four sections
of the piece. The sound should be generally dry, although some pedal should be
used when appropriate for phrasing and blending, especially on tremolo
passages. The sustain pedal should remain fully depressed for the entirety of
section D. Inside-piano glissandi are notated proportional to the lower
interior portion of the instrument -- from the lowest string up to the first
cross-bar -- and should be played with the fingertip.

Percussion should use hard sticks on toms, woodblocks, snares and crotales, and
softer mallets on marimba, bass drum and tam-tam (when ergonomic to do so).

All mordents and trills are a major 2nd, unless otherwise specified. All
tremolos are unmeasured and should evoke an even, cloud-like texture.

\end{textblock}

\end{document}